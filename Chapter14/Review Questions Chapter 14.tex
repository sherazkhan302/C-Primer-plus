\documentclass{amsart}

\usepackage{amssymb,latexsym, verbatim}
\thispagestyle{empty}
\pagestyle{empty}

\begin{document}
\begin{center}
	\Large {\bfseries
	\emph{C++ Primer Plus, $5^{\text{th}}$ Edition} by Stephen Prata \\
	Chapter 14: Reusing Code in C\raisebox{.15ex}{++} \\
	Review Questions} \normalsize \vspace{5ex}
\end{center}

% Note: in order to really get the formatting I want, I need to create my own environment. It would be similar to the enumerate environment, but instead of enclosing the enumerator in parentheses, there would just be a period after it. For example, rather than (1), we would have 1. instead.

\vfill
\noindent 1. 
\begin{minipage}[t]{11.5 cm}
	For each of the following sets of classes, indicate whether public 
	or private derivation is more appropriate for Column B: \\
	\begin{center}
		\begin{tabular}{l l}
			\hline
			A 		& B \\ \hline
			class Bear	& class PolarBear \\ \hline
			class Kitchen	& class Home \\ \hline
			class Person	& class Programmer \\ \hline
			class Person	& class HorseandJockey \\ \hline
			class Person, class Automobile & class Driver \\ \hline
		\end{tabular}
	\end{center}
\end{minipage}
\vfill
\phantom{3. } 
\begin{minipage}[t]{11.5 cm}
	{\slshape 
		\begin{itemize}
			\item 
				The PolarBear class should be derived publically since a
				polar bear is a kind of bear.
			\item
				The Home class should derive the Kitchen class privately since
				a home has a kitchen.
			\item
				The Programmer class should use public derivation since
				a programmer is a person.
			\item
				The HorseandJockey class should use private derivation since
				a horse and jockey have a person.
			\item
				The Driver class should use public derivation to inherit the
				Person class since a driver is a person. 
				This class should also use private derivation to inherit the 
				Automobile class since a driver has an automobile. 
		\end{itemize}
	} 
\end{minipage} 
\vfill
\newpage

\phantom{\quad}
\vfill
\noindent 2. 
\begin{minipage}[t]{11.5 cm}
	Suppose you have the following definitions:
	\begin{verbatim}
		class Frabjous {
		private:
		    char fab[20];
		public:
		    Frabjous(const char * s = "C++") : fab(s) { }
		    virtual void tell () { cout << fab; }
		};

		class Gloam {
		private:
		    int glip;
		    Frabjous fb;
		public:
		    Gloam(int g = 0, const char * s = "C++");
		    Gloam(int g, const Frabjous & f);
		    void tell();
		};
	\end{verbatim}
	Given that \texttt{Gloam} version of \texttt{tell()} should display the values of \texttt{glip} and \texttt{fb}, provide definitions for the three \texttt{Gloam} methods.
\end{minipage}

\vfill

\phantom{2. } 
\begin{minipage}[t]{11.5 cm}
	{\slshape 
		See the following code:
	}
	\begin{verbatim}
		Gloam::Gloam(int g, const char * s) : fb(s), glip(g)
		{
		}

		Gloam::Gloam(int g, const Frabjous & f) : fb(f), glip(g)
		{
		}

		void Gloam::tell()
		{
		   std::cout << "fb = ";
		   fb.tell();
		   std::cout << std::endl;
		   std::cout << "glip = " << glip << std::endl;
		}
	\end{verbatim} 
\end{minipage} 
\vfill
\newpage

\phantom{\quad}
\vfill
\noindent 3. 
\begin{minipage}[t]{11.5 cm}
	Suppose you have the following definitions:
	\begin{verbatim}
		class Frabjous {
		private:
		    char fab[20];
		public:
		    Frabjous(const char * s = "C++") : fab(s) { }
		    virtual void tell () { cout << fab; }
		};

		class Gloam : private Frabjous {
		private:
		    int glip;
		public:
		    Gloam(int g = 0, const char * s = "C++");
		    Gloam(int g, const Frabjous & f);
		    void tell();
		};
	\end{verbatim}
Given that \texttt{Gloam} version of \texttt{tell()} should display the values of \texttt{glip} and \texttt{fb}, provide definitions for the three \texttt{Gloam} methods.
\end{minipage}

\vfill

\phantom{3. } 
\begin{minipage}[t]{11.5 cm}
	{\slshape 
		See the following code:
	}
	\begin{verbatim}
		Gloam::Gloam(int g, const char * s) : Frabjous(s), glip(g)
		{
		}

		Gloam::Gloam(int g, const Frabjous & f) : Frabjous(f), glip(g)
		{
		}

		void Gloam::tell()
		{
		   std::cout << "fb = ";
		   Frabjous::tell();
		   std::cout << std::endl;
		   std::cout << "glip = " << glip << std::endl;
		}
	\end{verbatim} 
\end{minipage} 
\vfill
\newpage

\phantom{\quad}
\vfill
\noindent 4. 
\begin{minipage}[t]{11.5 cm}
	Suppose you have the following definition, based on the \texttt{Stack} template of Listing 14.13 and the \texttt{Worker} class of Listing 14.10:
	\begin{verbatim}
		Stack<Worker *> sw;
	\end{verbatim}
Write out the class declaration that will be generated. Just do the class declaration, not the non-inline class methods.
\end{minipage}
\vfill
\phantom{2. } 
\begin{minipage}[t]{11.5 cm}
	{\slshape 
		See the following code:
	} 
	\begin{verbatim}
		class Stack
		{
		private:
		   enum {MAX = 10}; 
		   Worker * items[MAX];
		   int top;
		public:
		   Stack();
		   bool isempty();
		   bool isfull();
		   bool push(const Worker * & item);
		   bool pop(Worker * & item);
		};
	\end{verbatim}
\end{minipage} 
\vfill

\noindent 5. 
\begin{minipage}[t]{11.5 cm}
	Use the template definitions in this chapter to define the following:
	\begin{itemize}
		\item An array of \texttt{string} objects
		\item A stack of arrays of \texttt{double}
		\item An array of stacks of pointers to \texttt{Worker} objects
	\end{itemize}
	How many template class definitions are produced in Listing 14.18?
\end{minipage}
\vfill
\phantom{3. } 
\begin{minipage}[t]{11.5 cm}
	\begin{itemize}
		\item
		   \verb+valarray<string> arr;+
		\item
		   \verb+stack< valarray<double> > stk;+
		\item
		   \verb+valarray< stack<Worker *> > arr;+
	\end{itemize}
	{\slshape
		Four template class definitions are produced in Listing 14.18.
	}
\end{minipage} 
\vfill

\noindent 6. 
\begin{minipage}[t]{11.5 cm}
	Describe the differences between virtual and nonvirtual base classes.
\end{minipage} \\[1ex]
\phantom{3. } 
\begin{minipage}[t]{11.5 cm}
	{\slshape 
		The difference arrises when a derived object inherits two 
		or more classes that were each derived from the same base 
		object.
		If we have non-virtual base classes, then the final object
		inherits as many copies of the original base class as it inherits 
		classes that were derived from the original base class.
		If we have virtual base classes, then the final derived
		class inherits a single copy of the original base class that
		is shared between all virtual base classes.
	}
\end{minipage} 
\vfill

\end{document}

Here is the format for questions that include multiple parts:

% Two parts 
\noindent X. 
\begin{minipage}[t]{11.5 cm}
	The question
\end{minipage} \\[1ex]
\phantom{1. }a.
\begin{minipage}[t]{11.5 cm}
	part a \\[1ex]
	{\slshape The answer.} \\
	{} % space for comments to be included
\end{minipage} \\[1ex]
\phantom{1. }b.
\begin{minipage}[t]{11.5 cm}
	part b \\[1ex]
	{\slshape The answer}\\
	{} % space for comments to be included
\end{minipage}
\\[2ex]

% Three Parts
\noindent X. 
\begin{minipage}[t]{11.5 cm}
	The question
\end{minipage} \\[1ex]
\phantom{1. }a.
\begin{minipage}[t]{11.5 cm}
	part a \\[1ex]
	{\slshape The answer.} \\
	{} % space for comments to be included
\end{minipage} \\[1ex]
\phantom{1. }b.
\begin{minipage}[t]{11.5 cm}
	part b \\[1ex]
	{\slshape The answer}\\
	{} % space for comments to be included
\end{minipage} \\[1ex]
\phantom{1. }c.
\begin{minipage}[t]{11.5 cm}
	part c \\[1ex]
	{\slshape The answer} \\
	{} % space for comments to be included
\end{minipage}
\\[2ex]

% Four Parts
\noindent X. 
\begin{minipage}[t]{11.5 cm}
	The question
\end{minipage} \\[1ex]
\phantom{1. }a.
\begin{minipage}[t]{11.5 cm}
	part a \\[1ex]
	{\slshape The answer.} \\
	{} % space for comments to be included
\end{minipage} \\[1ex]
\phantom{1. }b.
\begin{minipage}[t]{11.5 cm}
	part b \\[1ex]
	{\slshape The answer}\\
	{} % space for comments to be included
\end{minipage} \\[1ex]
\phantom{1. }c.
\begin{minipage}[t]{11.5 cm}
	part c \\[1ex]
	{\slshape The answer} \\
	{} % space for comments to be included
\end{minipage} \\[1ex]
\phantom{1. }d.
\begin{minipage}[t]{11.5 cm}
	part d \\[1ex]
	{\slshape The answer} \\
	{} % space for comments to be included
\end{minipage}
\\[2ex]

% Five Parts
\noindent X. 
\begin{minipage}[t]{11.5 cm}
	The question
\end{minipage} \\[1ex]
\phantom{1. }a.
\begin{minipage}[t]{11.5 cm}
	part a \\[1ex]
	{\slshape The answer.} \\
	{} % space for comments to be included
\end{minipage} \\[1ex]
\phantom{1. }b.
\begin{minipage}[t]{11.5 cm}
	part b \\[1ex]
	{\slshape The answer}\\
	{} % space for comments to be included
\end{minipage} \\[1ex]
\phantom{1. }c.
\begin{minipage}[t]{11.5 cm}
	part c \\[1ex]
	{\slshape The answer} \\
	{} % space for comments to be included
\end{minipage} \\[1ex]
\phantom{1. }d.
\begin{minipage}[t]{11.5 cm}
	part d \\[1ex]
	{\slshape The answer} \\
	{} % space for comments to be included
\end{minipage} \\[1ex]
\phantom{1. }e.
\begin{minipage}[t]{11.5 cm}
	part e \\[1ex]
	{\slshape The answer} \\
	{} % space for comments to be included
\end{minipage}
\\[2ex]

% Six Parts
\noindent X. 
\begin{minipage}[t]{11.5 cm}
	The question
\end{minipage} \\[1ex]
\phantom{1. }a.
\begin{minipage}[t]{11.5 cm}
	part a \\[1ex]
	{\slshape The answer.} \\
	{} % space for comments to be included
\end{minipage} \\[1ex]
\phantom{1. }b.
\begin{minipage}[t]{11.5 cm}
	part b \\[1ex]
	{\slshape The answer}\\
	{} % space for comments to be included
\end{minipage} \\[1ex]
\phantom{1. }c.
\begin{minipage}[t]{11.5 cm}
	part c \\[1ex]
	{\slshape The answer} \\
	{} % space for comments to be included
\end{minipage} \\[1ex]
\phantom{1. }d.
\begin{minipage}[t]{11.5 cm}
	part d \\[1ex]
	{\slshape The answer} \\
	{} % space for comments to be included
\end{minipage} \\[1ex]
\phantom{1. }e.
\begin{minipage}[t]{11.5 cm}
	part e \\[1ex]
	{\slshape The answer} \\
	{} % space for comments to be included
\end{minipage} \\[1ex]
\phantom{1. }f.
\begin{minipage}[t]{11.5 cm}
	part f \\[1ex]
	{\slshape The answer} \\
	{} % space for comments to be included
\end{minipage}
\\[2ex]

\noindent X. 
\begin{minipage}[t]{11.5 cm}
	The question
\end{minipage} \\[1ex]
\phantom{3. } 
\begin{minipage}[t]{11.5 cm}
	{\slshape The answer.} 
\end{minipage} 
\\[2ex]

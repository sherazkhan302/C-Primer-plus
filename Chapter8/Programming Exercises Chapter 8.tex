\documentclass[10 pt]{amsart}

\usepackage{amssymb,latexsym}
\usepackage{graphicx}

% For the cpart environment, although it would probably be better in the
% future to implement this with a list environment.
\newlength{\cgap}
\settowidth{\cgap}{\qquad \textbf{x. }}
\newlength{\cwidth}
\setlength{\cwidth}{\textwidth}
\addtolength{\cwidth}{-\cgap}
\newenvironment{cpart}[2][\cwidth]
	{\\ \phantom{\qquad}\textbf{#2. }\begin{minipage}[t]{#1}}
	{\end{minipage}}

\newcommand{\ttt}[1]{\texttt{#1}}
\newcommand{\ttb}[1]{\pmb{\texttt{#1}}}
% Macros, all must be filled out
\newcommand{\ChapNum}{8}

\begin{document}

	\title
	[Chapter \ChapNum]
	{C++ Primer Plus, 5$^\text{th}$ Edition \\
	Programming Exercises \\
	Chapter \ChapNum}

	\maketitle

	\begin{cpart}{1}
		Write a function that normally takes one argument, the 
		address of a string, and prints that string once.
		However, if a second type, type \ttt{int}, argument is 
		provided and is nonzero, the function should print the
		string a number of times equal to the number of times that the
		function has been called at that point.
		(Note that the number of times the string is printed is not
		equal to the value of the second argument, it is equal
		to the number of times teh function has been called.)
		Yes, this is a silly function, but it makes you use some of the
		techniques discussed in this chapter.
		Use the function in a simple program that demonstrates how the
		function works.
	\end{cpart}
	\vspace{2ex}

	\begin{cpart}{2}
		The \ttt{CandyBar} structure contains three members.
		The first member holds the brand name of a candy bar.
		The second member holds the weight (which may have a fractional
		value) of the candy bar, and the third member holds the number
		of calories (an integer value) in the candy bar.
		Write a program that uses a function that takes as arguments
		a reference to \ttt{CandyBar}, a pointer-to-\ttt{char}, a 
		\ttt{double}, and an \ttt{int} and uses the last three
		values to set the corresponding members of the structure.
		The last three arguments should have default values of 
		"Millennium Munch", 2.85, and 350.
		Also, the program should use a function that takes a reference
		to a \ttt{CandyBar} as an argument and displays
		the contents of the structure.
		Use \ttt{const} where appropriate.
	\end{cpart}
	\vspace{2ex}

	\begin{cpart}{3}
		Write a function that takes a reference to a \ttt{string}
		object as its parameter and that converts the contents
		of the \ttt{string} to uppercase.
		Use the \ttt{toupper()} function described in table 6.4.
		Write a program that uses a loop which allows you to test
		the function with different input.
		A sample run might look like this:
		{\ttfamily
			\begin{tabbing}
				\phantom{\qquad}\=\phantom{\qquad}\=\phantom{\qquad}\= \\
				Enter a string (q to quit):\ttb{ go away} \\
				GO AWAY \\
				Next string (q to quit):\ttb{ good grief!} \\
				GOOD GRIEF! \\
				Next string (q to quit):\ttb{ q} \\
				Bye.
			\end{tabbing}
		}
	\end{cpart}
	\vspace{2ex}

	\begin{cpart}{4}
		The following is a program skeleton:
		{\ttfamily
			\begin{tabbing}
				\phantom{\qquad}\=\phantom{\qquad}\=\hspace{3 cm}\= \\
				\#include <iostream> \\
				using namespace std; \\
				\#include <cstring> \\
				struct stringy \{ \\
				\> 	char * str; \\
				\> 	int ct;
				\}; \\
				\\
				// prototypes for set(), show(), and show() go here \\
				int main() \\
				\{ \\
				\> 	stringy beany; \\
				\> 	char testing[] = "Reality isn't what it used to be."; \\
				\\
				\> 	set(beany, testing); \qquad // first
														argument is a reference, \\
				\> \> // allocates space to hold copy of testing, \\
				\> \> // sets str member of beany to point to the \\
				\> \> // new block, copies testing to new block, \\
				\> \> // and sets ct member of beany \\
				\> 	show(beany); \> \> // prints member string once \\
				\> 	show(beany, 2); \> \> // prints member string twice \\
				\> 	testing[0] = 'D'; \\
				\> 	testing[1] = 'u'; \\
				\> 	show(testing); \> \> // prints testing string once \\
				\> 	show(testing, 3); \> \> // prints testing string thrice \\
				\> 	show("Done!"); \\
				\> 	return 0; \\
				\}
			\end{tabbing}
		}
		\phantom{\quad} \\
		Complete this skeleton by providing the described functions
		and prototypes.
		Note that there should be two \ttt{show()} functions, each
		using default arguments.
		Use \ttt{const} arguments when appropriate.
		Note that \ttt{set()} should use \ttt{new} to allocate
		sufficient space to hold the designated string.
		The techniques used here are similar to those used in designing
		and implementing classes.
		(You might have to alter the header filenames and delete
		the \ttt{using} directive, depending on your compiler.)
	\end{cpart}
	\vspace{2ex}

	\begin{cpart}{5}
		Write a template fucntion \ttt{max5()} that takes as its
		argument an array of five items of type \ttt{T} and
		returns the largest item in the array.
		(Because the size is fixed, it can be hard-coded into the loop
		instead of being passed as an argument.)
		Test it in a program that uses the function with an array
		of five \ttt{int} value and an array of five \ttt{double}
		values.
	\end{cpart}
	\vspace{2ex}

	\begin{cpart}{6}
		Write a template function \ttt{maxn()} that takes as its 
		arguments an array of items of type \ttt{T} and an 
		integer representing the number of elements in the array and
		that returns teh largest item in the array.
		Test it in a program that uses the function template with
		an array of six \ttt{int} values and an array of four
		\ttt{double} values.
		The program should also include a specialization that takes
		an array of pointers-to-\ttt{char} as an argument and the
		number of pointers as a second argument and that returns
		the address of the longest string.
		If multiple strings are tied for having the longest length,
		the function should return the address of the first one tied
		for longest.
		Test the specialization with an array of five string pointers.
	\end{cpart}
	\vspace{2ex}

	\begin{cpart}{7}
		Modify Listing 8.14 so that the template functions return
		the sum of the array contents instead of displaying the 
		contents.
		The program should now report the total number of things
		and the sum of all the debts.
	\end{cpart}
	\vspace{2ex}

\end{document}

regarding tabbing environments:
\= (set tab)
\> (advance to next tab stop)
\<
\+ (indent; move margin right)
\- (unindent; move margin left)
\'
\`
\\ (end of line; newline)
\kill (ignore preceding text; use only for spacing)



{\ttfamily
	\begin{tabbing}
		\phantom{\qquad}\=\phantom{\qquad}\=\phantom{\qquad}\= \\
		
	\end{tabbing}
}












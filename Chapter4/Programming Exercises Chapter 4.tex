\documentclass[10 pt]{amsart}

\usepackage{amssymb,latexsym}
\usepackage{graphicx}

% For the cpart environment, although it would probably be better in the
% future to implement this with a list environment.
\newlength{\cgap}
\settowidth{\cgap}{\qquad \textbf{x. }}
\newlength{\cwidth}
\setlength{\cwidth}{\textwidth}
\addtolength{\cwidth}{-\cgap}
\newenvironment{cpart}[2][\cwidth]
	{\\ \phantom{\qquad}\textbf{#2. }\begin{minipage}[t]{#1}}
	{\end{minipage}}

% Macros, all must be filled out
\newcommand{\ChapNum}{4}

\begin{document}

	\title
	[Chapter \ChapNum]
	{C++ Primer Plus, 5$^\text{th}$ Edition \\
	Programming Exercises \\
	Chapter \ChapNum}

	\maketitle

	\begin{cpart}{1}
		Write a C++ program that requests and displays information
		as shown in the following example of output:
		{\ttfamily
			\begin{tabbing}
				\phantom{\qquad}\=\phantom{\qquad}\=\phantom{\qquad}\= \\
				What is your first name? {\bf Betty Sue}}\\
				What is your last name? {\bf Yew} \\
				What letter grade do you deserve? {\bf B}\\ 
				What is your age? {\bf 22} \\
				Name: Yew, Betty Sue \\
				Grade: C \\
				Age: 22
			\end{tabbing}
		}
		Note that the program should be able to accept first names
		that comprise more than one word.
		Also note that the program adjusts the grade downward--that is,
		up one letter.
		Assume that the user requests an A, B, or C so that you don't
		have to worry about the gap between D and F.
	\end{cpart}
	\vspace{2ex}

	\begin{cpart}{2}
		Rewrite Listing 4.4, using the C++ \texttt{string} class
		instead of \texttt{char} arrays.
	\end{cpart}
	\vspace{2ex}

	\begin{cpart}{3}
		Write a program that asks the user to enter his or her first
		name and then last name, and that then constructs, stores, 
		and displays a third string, consisting of the user's last name
		followed by a comma, a space, and first name. 
		Use \texttt{char} arrays and functions from the \texttt{cstring}
		header file. 
		A sample run could look like this:
		{\ttfamily
			\begin{tabbing}
				\phantom{\qquad}\=\phantom{\qquad}\=\phantom{\qquad}\= \\
				Enter your first name: {\bf Flip} \\
				Enter your last name: {\bf Fleming} \\
				Here's the information in a single string:
					Fleming, Flip
			\end{tabbing}
		}		
	\end{cpart}
	\vspace{2ex}

	\begin{cpart}{4}
		Write a program that asks the user to enter his or her first 
		name and then last name, and that then constructs, stores, 
		and displays a third string consisting of the user's last name
		followed by a comma, a space, and first name.
		Use \texttt{string} objects and methods from the \texttt{string}
		header file.
		A sample run could look like this:
		{\ttfamily
			\begin{tabbing}
				\phantom{\qquad}\=\phantom{\qquad}\=\phantom{\qquad}\= \\
				Enter your first name: {\bf Flip} \\
				Enter your last name: {\bf Fleming} \\
				Here's the information in a single string:
					Fleming, Flip
			\end{tabbing}
		}		
	\end{cpart}
	\vspace{2ex}

	\begin{cpart}{5}
		The \texttt{CandyBar} structure contains three members.
		The first member holds the brand name of a candy bar.
		The second member holds the weight (which may have a fractional
		part) of the candy bar, and the third member holds the number
		of calories (an integer value) in the candy bar.
		Write a program that declares such a structure and creates a
		\texttt{CandyBar} variable called \texttt{snack}, initializing
		its members to \texttt{"Mocha Munch"}, \texttt{2.3}, and 
		\texttt{350}, respectively. 
		The initialization should be part of the declaration for 
		\texttt{snack}.
		Finally, the program should display the contents of the 
		\texttt{snack} variable.
	\end{cpart}
	\vspace{2ex}

	\begin{cpart}{6}
		The \texttt{CandyBar} structure contains three members, as 
		described in Programming Exercise 5.
		Write a program that creates an array of three \texttt{CandyBar}
		structures, initializes them to values of your choice, and
		then displays the contents of each structure.
	\end{cpart}
	\vspace{2ex}

	\begin{cpart}{7}
		William Wingate runs a pizza-analysis service. 
		For each pizza, he needs to record the following information:
		\begin{itemize}
			\item The name of the pizza company, which can consist of
				more than one word 
			\item The diameter of the pizza
			\item The weight of the pizza
		\end{itemize}
		Devise a structure that can hold this information and write a
		program that uses a structure variable of that type.
		The program should ask the user to enter each of the preceeding
		items of information, and then the program should display that
		information. 
		Use \texttt{cin} (or its methods) and \texttt{cout}.
	\end{cpart}
	\vspace{2ex}

	\begin{cpart}{8}
		Do Programming Exercise 7, but use \texttt{new} to allocate a 
		structure instead of declaring a structure variable.
		Also, have the program request the pizza diameter before
		it requests the pizza company name.
	\end{cpart}

	\begin{cpart}{9}
		Do Programming Exercise 6, but, instead of declaring an array
		of three \texttt{CandyBar} structures, use \texttt{new} to 
		allocate the array dynamically.
	\end{cpart}

\end{document}

regarding tabbing environments:
\= (set tab)
\> (advance to next tab stop)
\<
\+ (indent; move margin right)
\- (unindent; move margin left)
\'
\`
\\ (end of line; newline)
\kill (ignore preceding text; use only for spacing)



{\ttfamily
	\begin{tabbing}
		\phantom{\qquad}\=\phantom{\qquad}\=\phantom{\qquad}\= \\
		
	\end{tabbing}
}












\documentclass{amsart}

\usepackage{amssymb,latexsym, verbatim}
\thispagestyle{empty}
\pagestyle{empty}

\begin{document}
\begin{center}
	\Large {\bfseries
	\emph{C++ Primer Plus, $5^{\text{th}}$ Edition} by Stephen Prata \\
	Chapter 5: Loops and Relational Expressions \\
	Review Questions} \normalsize \vspace{.5 cm}
\end{center}

% Note: in order to really get the formatting I want, I need to create my own environment. It would be similar to the enumerate environment, but instead of enclosing the enumerator in parentheses, there would just be a period after it. For example, rather than (1), we would have 1. instead.

% Note 8.23.11: Also, stop forgetting to indent in environments. It's important that my code is readable. 

% Note 8.23.11: I realized. If I'm going to be making a template, I should try to format it with care and then test-run it on at least two documents to make sure that I'm not missing anything. In this case, I discovered that I should have used relative, not absolute spacing, and that extra spacing after each question makes the document look nicer. Now I can't fix everything without doing it manually :(

\noindent 1. 
\begin{minipage}[t]{11.5 cm}
	What's the difference between an entry-condition loop and an exit-condition loop? Which kind is each of the C\raisebox{0.15ex}{++} loops?
\end{minipage} \\[1ex]
\phantom{3. } 
\begin{minipage}[t]{11.5 cm}
	{\slshape An entry-condition loop evaluates the test-condition before the first execution of the body. An exit-condition loop evaluates the test-condition after the first execution of the body. Thus, an entry-condition loop executing at least once is contingent on the test-condition evaluating to \texttt{true} the first time and an exit-conditon loop will always execute the body at least once, even if the test-condition is false. \texttt{for} and \texttt{while} loops are entry-condition loops and \texttt{do while} loops are exit-condition loops.} 
\end{minipage} 
\\[.2cm]

\noindent 2. 
\begin{minipage}[t]{11.5 cm}
	What would the following code fragment print if it were part of a valid program?
\begin{verbatim}
	int i;
	for (i = 0; i < 5; i++)
	      cout << i;
	      cout << endl;
\end{verbatim}
\end{minipage} \\[1ex]
\phantom{2. } 
\begin{minipage}[t]{11.5 cm}
	{\slshape The code would print the following:}
	\begin{verbatim}
		01234
	\end{verbatim} 
\end{minipage} 
\\[.2cm]

\noindent 3. 
\begin{minipage}[t]{11.5 cm}
	What would the following code fragment print if it were part of a valid program?
\begin{verbatim}
	int j;
	for(j = 0; j < 11; j +=3)
	      cout << j;
	cout << endl << j << endl;
\end{verbatim}
\end{minipage} \\[1ex]
\phantom{3. } 
\begin{minipage}[t]{11.5 cm}
	{\slshape The code would print the following:}
	\begin{verbatim}
		0369
		12
	\end{verbatim} 
\end{minipage} 
\\[.2cm]

\noindent 4. 
\begin{minipage}[t]{11.5 cm}
	What would the following code fragment print if it were part of a valid program?
\begin{verbatim}
	int j = 5;
	while (++j < 9)
	      cout << j++ << endl;
\end{verbatim}
\end{minipage} \\[1ex]
\phantom{2. } 
\begin{minipage}[t]{11.5 cm}
	{\slshape The code would print the following:}
	\begin{verbatim}
		6
		8
	\end{verbatim} 
\end{minipage} 
\\[.2cm]

\noindent 5. 
\begin{minipage}[t]{11.5 cm}
	What would the following code fragment print if it were part of a valid program?
\begin{verbatim}
	int k = 8;
	do
	      cout << "k = " << k << endl;
	while (k++ < 5);
\end{verbatim}
\end{minipage} \\[1ex]
\phantom{3. } 
\begin{minipage}[t]{11.5 cm}
	{\slshape The code would print the following:}
	\begin{verbatim}
		k = 8
	\end{verbatim} 
\end{minipage} 
\\[.2cm]

\noindent 6. 
\begin{minipage}[t]{11.5 cm}
	Write a \texttt{for} loop that prints the values 1 2 4 8 16 32 64 by increasing the value of a counting variable by a factor of two in each cycle. 
\end{minipage} \\[1ex]
\phantom{3. } 
\begin{minipage}[t]{11.5 cm}
	{\slshape See the following code:}
	\begin{verbatim}
		for (int i = 1; i < 65; i *= 2)
		    cout << i << " ";
	\end{verbatim}
\end{minipage} 
\\[.2cm]

\noindent 7. 
\begin{minipage}[t]{11.5 cm}
	How do you make a loop body include more than one statement?
\end{minipage} \\[1ex]
\phantom{2. } 
\begin{minipage}[t]{11.5 cm}
	{\slshape To make a loop body include more than one statement, braces should immediately follow the loop and enclose whichever statements are to be included in the body of the loop.} 
\end{minipage} 
\\[.2cm]

\noindent 8. 
\begin{minipage}[t]{11.5 cm}
	Is the following statement valid? If not, why not? If so, what does it do? \\[1ex]
\verb+int x = (1,024);+ \\[1 ex]
What about the following?
\begin{verbatim}
	int y;
	y = 1,024;
\end{verbatim}
\end{minipage} \\[1ex]
\phantom{3. } 
\begin{minipage}[t]{11.5 cm}
	{\slshape Both statements are valid. The first statement assigns the value 024 to \texttt{x}. Since 024 is octal, it is converted to an int which would be 20 in decimal. The second statement initializes \texttt{y} to 1. The comma operator separates two or more expressions where only one is expected. The comma operator has the lowest precedence among any C++ operators. When the set of expressions has to be evaluated for a value, only the rightmost expression is considered. } 
\end{minipage} 
\\[.2cm]

\noindent 9. 
\begin{minipage}[t]{11.5 cm}
	How does \verb+cin>>ch+ differ from \texttt{cin.get(ch)} and \texttt{ch=cin.get()} and how it views input?
\end{minipage} \\[1ex]
\phantom{2. } 
\begin{minipage}[t]{11.5 cm}
	{\slshape \verb+cin>>ch+ skips over whitespace, where the other two do not.} 
\end{minipage} 

\end{document}

Here is the format for questions that include multiple parts:

\noindent X. 
\begin{minipage}[t]{11.5 cm}
	The question
\end{minipage} \\
\phantom{1. }
\begin{minipage}[t]{11.5 cm}
	a. part a \\
	\phantom{a. } {\slshape The answer. You can also type code: \verb+cout << "Hello" << endl;+} \\
	{} % space for comments to be included
\end{minipage} \\
\phantom{1. }
\begin{minipage}[t]{11.5 cm}
	b. part b \\
	\phantom{b. } {\slshape The answer} \\
	{} % space for comments to be included
\end{minipage} \\
\phantom{1. }
\begin{minipage}[t]{11.5 cm}
	c. part c \\
	\phantom{c. } {\slshape The answer} \\
	{} % space for comments to be included
\end{minipage} \\
\phantom{1. }
\begin{minipage}[t]{11.5 cm}
	d. part d \\
	\phantom{d. } {\slshape The answer} \\
	{} % space for comments to be included
\end{minipage}
\\[.2cm]

\noindent X. 
\begin{minipage}[t]{11.5 cm}
	The question
\end{minipage} \\
\phantom{3. } 
\begin{minipage}[t]{11.5 cm}
	{\slshape The answer.} 
\end{minipage} 
\\[.2cm]

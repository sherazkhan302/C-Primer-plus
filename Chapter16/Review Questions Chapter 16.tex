\documentclass{amsart}

\usepackage{amssymb,latexsym, verbatim}
\thispagestyle{empty}
\pagestyle{empty}

\begin{document}
\begin{center}
	\Large {\bfseries
	\emph{C++ Primer Plus, $5^{\text{th}}$ Edition} by Stephen Prata \\
	Chapter 16: The \texttt{string} Class and the Standard Template Library \\
	Review Questions} \normalsize \vspace{5ex}
\end{center}

% Note: in order to really get the formatting I want, I need to create my own environment. It would be similar to the enumerate environment, but instead of enclosing the enumerator in parentheses, there would just be a period after it. For example, rather than (1), we would have 1. instead.

\phantom{\quad}\vfill
\noindent 1. 
\begin{minipage}[t]{11.5 cm}
	Consider the following class declaration:
	\begin{verbatim}
		class RQ1
		{
		private:
		    char * st;     // points to C-style string
		public:
		    RQ1() { st = new char [1]; strcpy(st,""); }
		    RQ1(const char * s)
		    {st = new char [strlen(s) + 1]; strcpy(st,s); }
		    RQ1(const RQ1 & rq)
		    {st = new char [strlen(rq.st) + 1]; strcpy(st,rq.st); }
		    ~RQ1() {delete [] st};
		    RQ & operator=(const RQ & rq);
		    // more stuff
		};
	\end{verbatim}
	Convert this to a declaration that uses a \texttt{string} object instead. What methods no longer need explicit definitions?
\end{minipage} \\[1ex]
\phantom{3. } 
\begin{minipage}[t]{11.5 cm}
	{\slshape 
		This is our class rewritten to use a \verb+string+ object.
	\begin{verbatim}
		class RQ1
		{
		private:
		    string st;
		public:
		    RQ1() { st = ""; }
		    RQ1(const char * s)
		    {st = string(s); }
		    RQ1(const RQ1 & rq)
		    {st = rq.st; }
		    ~RQ1() {}
		    RQ & operator=(const RQ & rq)
		    {st = rq.st; return *this;}
		    // more stuff
		};
	\end{verbatim}
	Barring the inconsistency regarding accessing the private member
	of another \verb+RQ1+ object directly, we have the following 
	results: 
	\begin{enumerate}
		\item
			We do not need an explicit copy constructor.
		\item
			We do not need an explicit destructor
		\item 
			We do not need an explicit overloaded assignment operator
	\end{enumerate}
	However, since we explicitly define a constructor, we are
	unable to use the default copy constructor, so to enable it
	we must define it explicitly.
	} 
\end{minipage} 
\vfill
\newpage

\phantom{\quad}\vfill
\noindent 2. 
\begin{minipage}[t]{11.5 cm}
	Name at least two advantages \texttt{string} objects have over C-style strings in terms of ease-of-use.
\end{minipage} \\[1ex]
\phantom{2. } 
\begin{minipage}[t]{11.5 cm}
	{\slshape 
		\begin{enumerate}
			\item
				We can change the size of a \verb+string+ object.
			\item
				We can use the assigment operator to assign string values.
			\item
				We can use the addition operator for easy concatenation 
				of two strings.
			\item 
				We can use relational operators (\verb+==, <, <=, >, >=+)
				on string objects for easy comparisons.
			\item
				The \verb+string+ class has member functions, such as 
				\verb+size()+ which allows us to work with \verb+string+
				objects in an intuitive way.
		\end{enumerate}
	} 
\end{minipage} 
\vfill

\noindent 3. 
\begin{minipage}[t]{11.5 cm}
	Write a function that takes a reference to a \texttt{string} object as an argument and that converts the \texttt{string} object to all uppercase.
\end{minipage} \\[1ex]
\phantom{3. } 
\begin{minipage}[t]{11.5 cm}
	{\slshape 
		\begin{verbatim}
			string & ToUpper(string & s)
			{
			   transform(s.begin(), s.end(), s.begin(), toupper);
			   return s;
			}
		\end{verbatim}
	} 
\end{minipage} 
\vfill

\noindent 4. 
\begin{minipage}[t]{11.5 cm}
	Which of the following are not examples of correct usage (conceptually or syntactically) of \verb+auto_ptr+? (Assume that the needed header files have been included.)
	\begin{verbatim}
		auto_ptr<int> pia(new int[20]);
		auto_ptr<string> (new string);
		int rigue = 7;
		auto_ptr<int> pr(&rigue);
		auto_ptr dbl (new double);
	\end{verbatim}
\end{minipage} \\[1ex]
\phantom{2. } 
\begin{minipage}[t]{11.5 cm}
	{\slshape 
		\begin{itemize}
			\item
				The first line of code is an incorrect usage because
				the destructor for \verb+auto_ptr<>+ uses the \verb+delete+
				command in its destructor, not \verb+delete []+.
			\item
				The fourth line of code is incorrect because when the destructor
				to \verb+pr+ is invoked, it would apply the \verb+delete+ operator
				to non-heap memory.
			\item
				The fifth line of code is incorrect because we must declare
				a specialization of \verb+auto_ptr+ since it is a template class.
		\end{itemize}
	} 
\end{minipage} 
\vfill 
\newpage

\phantom{\quad}\vfill
\noindent 5. 
\begin{minipage}[t]{11.5 cm}
	If you could make the mechanical equivalent of a stack that held golf clubs instead of numbers, why would it (conceptually) be a bad golf bag?
\end{minipage} \\[1ex]
\phantom{3. } 
\begin{minipage}[t]{11.5 cm}
	{\slshape 
		It would be a bad golf bag because you would be unable to access
		any given golf club easily. 
		For example, if you have nine golf clubs and needed to get 
		the \#3 club, firstly you would not know where in the stack
		the club was unless you remember the order you placed all
		of the clubs in the bag.
		Secondly, to access the \#3 club, you would need to remove
		each club placed onto the stack after the \#3 club to get to it.
	} 
\end{minipage} 
\vfill

\noindent 6. 
\begin{minipage}[t]{11.5 cm}
	Why would a \texttt{set} container be a poor choice for storing a hole-by-hole record of your golf scores?
\end{minipage} \\[1ex]
\phantom{3. } 
\begin{minipage}[t]{11.5 cm}
	{\slshape 
		The best reason for why this would be a poor choice is that
		a \verb+set+ object does not hold duplicate items.
		If you were to get the same score on two holes, the
		set would count them as equivalent (if the only thing stored
		is the score).
		Additionally, a \verb+set+ object automatically orders its 
		elements by using the \verb+<+ operator.
		If the only thing stored was the numeric score per hole,
		you would be unable to tell which score corresponded to which
		hole. 
	}
\end{minipage} 
\vfill

\noindent 7. 
\begin{minipage}[t]{11.5 cm}
	Because a pointer is an iterator, why didn't the STL designers simply use pointers instead of iterators?
\end{minipage} \\[1ex]
\phantom{2. } 
\begin{minipage}[t]{11.5 cm}
	{\slshape 
		Pointers are not appropriate for all situations.
		In the case of an array, a pointer will work fine as it 
		can be dereferenced, incremented, and random access is 
		permissible. 
		However, for some data structures, such as a linked list,
		a binary tree, and a heap, pointers themselves
		will not work the same way when incremented and 
		may not allow random access.
		However, an iterator can be made so that operators such 
		as incrementation work in the same way for a linked list
		as it does for an array.
		In short, pointers are only appropriate for a subset of
		data structures. 
	} 
\end{minipage} 
\vfill
\newpage

\phantom{\quad}\vfill
\noindent 8. 
\begin{minipage}[t]{11.5 cm}
	Why didn't the STL designers simply define a base iterator class, use inheritance to derive classes for the other iterator types, and express the algorithms in terms of those iterator classes?
\end{minipage} \\[1ex]
\phantom{3. } 
\begin{minipage}[t]{11.5 cm}
	{\slshape 
		I suppose that to do so would be possible. 
		However, because of the multitude of data structures, 
		each of which requiring unique implementations of
		the iterator concept, the iterator member functions
		would have to be redefined for each iterator.
		In this case, inheritance would be in name only
		which defeats the purpose of inheritance.
		With this in mind, using inheritance seems to complicate
		the implementation of an iterator for a given class
		without any sort of benefit.
	} 
\end{minipage} 
\vfill

\noindent 9. 
\begin{minipage}[t]{11.5 cm}
	Give at least three examples of convenience advantages that a \texttt{vector} object has over an ordinary array.
\end{minipage} \\[1ex]
\phantom{2. } 
\begin{minipage}[t]{11.5 cm}
	{\slshape 
		\begin{enumerate}
			\item
				We can use STL functions with the \verb+vector<>+ class
			\item
				We can resize a \verb+vector<>+ object easily
			\item
				We can use iterators to initialize a \verb+vector<>+ object
			\item
				The \verb+vector<>+ class has member functions for
				performing many operations.
		\end{enumerate}
	} 
\end{minipage} 
\vfill

\noindent 10. 
\begin{minipage}[t]{11.5 cm}
	If Listing 16.7 were implemented with \texttt{list} instead of \texttt{vector}, what parts of the program would become invalid? Could the invalid part be fixed easily? If so, how?
\end{minipage} \\[1ex]
\phantom{3. } 
\begin{minipage}[t]{11.5 cm}
	{\slshape 
		\begin{enumerate}
			\item
				The STL sort requires a random access iterator, the 
				\verb+list<>+ class only has bidirectional iterators, which
				do not allow for random access.
				However, this problem can be fixed by using the 
				\verb+sort()+ member function of the \verb+list<>+ class.
			\item
				The \verb+random_shuffle()+ function requires
				a random access iterator. 
				To remedy this, the client program must provide a function
				that shuffles the values of a linked list.
				However, this is not necessarily an easy fix.
		\end{enumerate}
	} 
\end{minipage} 
\vfill

\end{document}

Here is the format for questions that include multiple parts:

% Two parts 
\noindent X. 
\begin{minipage}[t]{11.5 cm}
	The question
\end{minipage} \\[1ex]
\phantom{1. }a.
\begin{minipage}[t]{11.5 cm}
	part a \\[1ex]
	{\slshape The answer.} \\
	{} % space for comments to be included
\end{minipage} \\[1ex]
\phantom{1. }b.
\begin{minipage}[t]{11.5 cm}
	part b \\[1ex]
	{\slshape The answer}\\
	{} % space for comments to be included
\end{minipage}
\\[2ex]

% Three Parts
\noindent X. 
\begin{minipage}[t]{11.5 cm}
	The question
\end{minipage} \\[1ex]
\phantom{1. }a.
\begin{minipage}[t]{11.5 cm}
	part a \\[1ex]
	{\slshape The answer.} \\
	{} % space for comments to be included
\end{minipage} \\[1ex]
\phantom{1. }b.
\begin{minipage}[t]{11.5 cm}
	part b \\[1ex]
	{\slshape The answer}\\
	{} % space for comments to be included
\end{minipage} \\[1ex]
\phantom{1. }c.
\begin{minipage}[t]{11.5 cm}
	part c \\[1ex]
	{\slshape The answer} \\
	{} % space for comments to be included
\end{minipage}
\\[2ex]

% Four Parts
\noindent X. 
\begin{minipage}[t]{11.5 cm}
	The question
\end{minipage} \\[1ex]
\phantom{1. }a.
\begin{minipage}[t]{11.5 cm}
	part a \\[1ex]
	{\slshape The answer.} \\
	{} % space for comments to be included
\end{minipage} \\[1ex]
\phantom{1. }b.
\begin{minipage}[t]{11.5 cm}
	part b \\[1ex]
	{\slshape The answer}\\
	{} % space for comments to be included
\end{minipage} \\[1ex]
\phantom{1. }c.
\begin{minipage}[t]{11.5 cm}
	part c \\[1ex]
	{\slshape The answer} \\
	{} % space for comments to be included
\end{minipage} \\[1ex]
\phantom{1. }d.
\begin{minipage}[t]{11.5 cm}
	part d \\[1ex]
	{\slshape The answer} \\
	{} % space for comments to be included
\end{minipage}
\\[2ex]

% Five Parts
\noindent X. 
\begin{minipage}[t]{11.5 cm}
	The question
\end{minipage} \\[1ex]
\phantom{1. }a.
\begin{minipage}[t]{11.5 cm}
	part a \\[1ex]
	{\slshape The answer.} \\
	{} % space for comments to be included
\end{minipage} \\[1ex]
\phantom{1. }b.
\begin{minipage}[t]{11.5 cm}
	part b \\[1ex]
	{\slshape The answer}\\
	{} % space for comments to be included
\end{minipage} \\[1ex]
\phantom{1. }c.
\begin{minipage}[t]{11.5 cm}
	part c \\[1ex]
	{\slshape The answer} \\
	{} % space for comments to be included
\end{minipage} \\[1ex]
\phantom{1. }d.
\begin{minipage}[t]{11.5 cm}
	part d \\[1ex]
	{\slshape The answer} \\
	{} % space for comments to be included
\end{minipage} \\[1ex]
\phantom{1. }e.
\begin{minipage}[t]{11.5 cm}
	part e \\[1ex]
	{\slshape The answer} \\
	{} % space for comments to be included
\end{minipage}
\\[2ex]

% Six Parts
\noindent X. 
\begin{minipage}[t]{11.5 cm}
	The question
\end{minipage} \\[1ex]
\phantom{1. }a.
\begin{minipage}[t]{11.5 cm}
	part a \\[1ex]
	{\slshape The answer.} \\
	{} % space for comments to be included
\end{minipage} \\[1ex]
\phantom{1. }b.
\begin{minipage}[t]{11.5 cm}
	part b \\[1ex]
	{\slshape The answer}\\
	{} % space for comments to be included
\end{minipage} \\[1ex]
\phantom{1. }c.
\begin{minipage}[t]{11.5 cm}
	part c \\[1ex]
	{\slshape The answer} \\
	{} % space for comments to be included
\end{minipage} \\[1ex]
\phantom{1. }d.
\begin{minipage}[t]{11.5 cm}
	part d \\[1ex]
	{\slshape The answer} \\
	{} % space for comments to be included
\end{minipage} \\[1ex]
\phantom{1. }e.
\begin{minipage}[t]{11.5 cm}
	part e \\[1ex]
	{\slshape The answer} \\
	{} % space for comments to be included
\end{minipage} \\[1ex]
\phantom{1. }f.
\begin{minipage}[t]{11.5 cm}
	part f \\[1ex]
	{\slshape The answer} \\
	{} % space for comments to be included
\end{minipage}
\\[2ex]

\noindent X. 
\begin{minipage}[t]{11.5 cm}
	The question
\end{minipage} \\[1ex]
\phantom{3. } 
\begin{minipage}[t]{11.5 cm}
	{\slshape The answer.} 
\end{minipage} 
\\[2ex]

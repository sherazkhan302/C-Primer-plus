\documentclass[10 pt]{amsart}

\usepackage{amssymb,latexsym}
\usepackage{graphicx}

% For the cpart environment, although it would probably be better in the
% future to implement this with a list environment.
\newlength{\cgap}
\settowidth{\cgap}{\qquad \textbf{x. }}
\newlength{\cwidth}
\setlength{\cwidth}{\textwidth}
\addtolength{\cwidth}{-\cgap}
\newenvironment{cpart}[2][\cwidth]
	{\\ \phantom{\qquad}\textbf{#2. }\begin{minipage}[t]{#1}}
	{\end{minipage}}

\newcommand{\ttt}[1]{\texttt{#1}}
\newcommand{\ttb}[1]{\pmb{\texttt{#1}}}
% Macros, all must be filled out
\newcommand{\ChapNum}{6}

\begin{document}

	\title
	[Chapter \ChapNum]
	{C++ Primer Plus, 5$^\text{th}$ Edition \\
	Programming Exercises \\
	Chapter \ChapNum}

	\maketitle

	\begin{cpart}{1}
		Write a program that reads keyboard input to the \@ symbol
		and that echoes the input except for digits, converting each 
		uppercase character to lowercase, and vice versa.
		(Don't forget the \ttt{cctype} family)
	\end{cpart}
	\vspace{2ex}

	\begin{cpart}{2}
		Write a program that reads up to 10 donation values into an 
		array of \ttt{double}.
		The program should terminate input on non-numeric input.
		It should report the average of the numbers and also report how
		many numbers in the array are larger than the average.
	\end{cpart}
	\vspace{2ex}

	\begin{cpart}{3}
		Write a precursor to a menu-driven program.
		The program should display a menu offering four choices,
		each labeled with a letter.
		If the user responds with a letter other than one of the four
		valid choices, the program should prompt the user to enter
		a vaoid response until the user complies.
		Then the program should use a switch to select a simple
		action based on the user's selection.
		A program run could look something like this:
		{\ttfamily
			\begin{tabbing}
				\hspace{5 cm}\=\phantom{\qquad}\=\phantom{\qquad}\= \\
				Please enter one of the following choices:\\
				c) carnivore \> p) pianist \\
				t) tree 		\> g) game \\
				Please enter a c, p, t, or g:\ttb{ q} \\
				Please enter a c, p, t, or g:\ttb{ t} \\
				A maple is a tree.
			\end{tabbing}
		}
	\end{cpart}
	\newpage

	\begin{cpart}{4}
		When you join the Benevolent Order of Programmers, you can
		be known at BOP meetings by your real name, your job title, 
		or your secret BOP name.
		Write a program that can list members by real name, by job
		title, by secret name, or by a member's preference.
		Base the program on the following structure.
		{\ttfamily
			\begin{tabbing}
				\phantom{\qquad}\=\hspace{5 cm}\=\phantom{\qquad}\= \\
				// Benevolent Order of Programmers name structure \\
				struct bop \{ \\
				\> 	char fullname[strsize]; \> // real name \\
				\>		char title[strsize]; \> // job title \\
				\> 	char bopname[strsize]; \> // secret BOP name \\
				\> 	int preference; \> 		
					// 0 = fullname, 1 = title, 2 = bopname \\
				\};
			\end{tabbing}
		}
		In the program, create a small array of such structures and
		initialize it to suitable values.
		Have the program run a loop that lets the user select from
		different alternatives:
		{\ttfamily
			\begin{tabbing}
				\hspace{5 cm}\=\phantom{\qquad}\= \\
				a. display by name \> b. display by title \\
				c. display by bopname \> d. display by preference \\
				q. quit
			\end{tabbing}
		}
		Note that "display by preference" does not mean display by
		preference member;
		it means display the member corresponding to the preference 
		number.
		For instance, if preference is \ttt{1}, choice \ttt{d}
		would display the programmer's job title. 
		A sample run may look something like the following:
		{\ttfamily
			\begin{tabbing}
				\hspace{5 cm}\=\phantom{\qquad}\= \\
				Benevolent Order of Programmers Report \\
				a. display by name \> b. display by title \\
				c. display by bopname \> d. display by preference \\
				q. quit \\
				Enter your choice:\ttb{ a} \\
				Wimp Macho \\
				Raki Rhodes \\
				Celia Laiter \\
				Hoppy Hipman \\
				Pat Hand \\
				Next choice:\ttb{ d} \\
				Wimp Macho \\
				Junior Programmer \\
				MIPS \\
				Analyst Trainee \\
				LOOPY \\
				Next choice:\ttb{ q} \\
				Bye!
			\end{tabbing}
		}
	\end{cpart}
	\vspace{2ex}

	\begin{cpart}{5}
		The Kingdom of Neutronia, where the unit of currency is the
		tvarp, has the following income tax code: \vspace{2ex} \\
		first 5,000 tvarps: 0\% tax \\[2ex]
		next 10,000 tvarps: 10\% tax \\[2ex]
		next 20,000 tvarps: 15\% tax \\[2ex]
		tvarps after 35,000: 20\% tax \\[2ex]
		For example, someone earning 38,000 tvarps would owe
		5,000 $\times$ 0.00 + 10,000 $\times$ 0.10 +
		20,000 $\times$ 0.15 + 3,000 $\times$ 0.20, or 4,600 tvarps.
		Write a program taht uses a loop to solicit incomes and to report
		tax owed.
		The loop should terminate when the suer enters a negative
		number or nonnumeric input.
	\end{cpart}
	\vspace{2ex}

	\begin{cpart}{6}
		Put together a program that keeps track of monetary contributions
		to the Society for the Preservation of Rightful Influence.
		It should ask the user to enter the number of contributors
		and then solicit the user to enter the name and contribution
		of each contributor.
		The information should be stored in a dynamically allocated array
		of structures.
		Each structure shoul dhave two members: a character array
		(or else a \ttt{string} object) to store the name and a
		\ttt{double} member to hold the amount of the contribution.
		After reading all the data, the program should display the names
		and amount donated for all donors who contributed \$10,000 or
		more.
		This list should be headed by the label Grand Patrons.
		After that, the program should list the remaining donors.
		That list should be headed Patrons.
		If there are no donors in one of the categories, the program
		should print the word "none."
		Aside from displaying two categories, the program need do
		no sorting.
	\end{cpart}
	\vspace{2ex}

	\begin{cpart}{7}
		Write a program that reads input a word at a time until a lone
		\ttb{q} is entered.
		The program shoudl then report the number of words that began with
		vowels, the number that began with consonants, and the number
		that fit neither of those categories.
		One approach is to use \ttt{isalpha()} to discriminate
		between words beginning with letters and those that don't  
		and then use an \ttt{if} or \ttt{switch} statement to further
		identify those passing the \ttt{isalpha()} test that begin with
		vowels.
		A sample run might look like this:
		{\ttfamily
			\begin{tabbing}
				\phantom{\qquad}\=\phantom{\qquad}\=\phantom{\qquad}\= \\
				Enter words (q to quit): \\
				\ttb{The 12 awesome oxen ambled} \\
				\ttb{quietly across 15 meters of lawn. q} \\
				5 words beginning with vowels \\
				4 words beginning with consonants \\
				2 others
			\end{tabbing}
		}
	\end{cpart}
	\vspace{2ex}

	\begin{cpart}{8}
		Write a program that opens a text file, reads it 
		character-by-character to the end of the file, and reports
		the number of characters in the file.
	\end{cpart}
	\vspace{2ex}

	\begin{cpart}{9}
		Do Programming Exercise 6, but modify it to get information
		from a file.
		The first item in the file should be the number of contributors,
		and the rest of the file should consist of pairs of lines,
		with the first line of each pair being a contributor's name
		and the second line being a contribution. 
		That is, the file should look like this:
		{\ttfamily
			\begin{tabbing}
				\phantom{\qquad}\=\phantom{\qquad}\=\phantom{\qquad}\= \\
				4\\
				Sam Stone \\
				2000 \\
				Freida Flass \\
				100500 \\
				Tammy Tubbs \\
				5000 \\
				Rich Raptor \\
				55000
			\end{tabbing}
		}
	\end{cpart}
	\vspace{2ex}

\end{document}

regarding tabbing environments:
\= (set tab)
\> (advance to next tab stop)
\<
\+ (indent; move margin right)
\- (unindent; move margin left)
\'
\`
\\ (end of line; newline)
\kill (ignore preceding text; use only for spacing)



{\ttfamily
	\begin{tabbing}
		\phantom{\qquad}\=\phantom{\qquad}\=\phantom{\qquad}\= \\
		
	\end{tabbing}
}












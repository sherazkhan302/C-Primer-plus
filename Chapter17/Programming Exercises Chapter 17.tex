\documentclass[10 pt]{amsart}

\usepackage{amssymb,latexsym}
\usepackage{graphicx, setspace, enumerate}
% the setspace package allows us use of the 
% spacing environment (\begin{spacing}{second arg}) where
% second arg is a number to multiply to the spacing factor.
% use 2 for double space, 1 for single space, etc.

% For the cpart environment, although it would probably be better in the
% future to implement this with a list environment.
\newlength{\cgap}
\settowidth{\cgap}{\textbf{x. }}
\newlength{\cwidth}
\setlength{\cwidth}{\textwidth}
\addtolength{\cwidth}{-\cgap}
\newenvironment{cpart}[2][\cwidth]
	{%		
		\\ %
		\textbf{#2. }%
		\begin{minipage}[t]{#1}%
		\setlength{\parindent}{0pt}%
		\setlength{\parskip}{2ex}%
	}
	{%
		\end{minipage}%
	}
\newenvironment{cpartContinued}[2][\cwidth]
	{%		
		\\ %
		\textbf{#2. (continued)}%
		\\
		\phantom{#2. }
		\begin{minipage}[t]{#1}%
		\setlength{\parindent}{0pt}%
		\setlength{\parskip}{2ex}%
	}
	{%
		\end{minipage}%
	}


% set paragraph spacing like that in the book
\setlength{\parindent}{0pt}
\setlength{\parskip}{2ex}

% these new commands will make typing different formats easier.
\newcommand{\ttt}[1]{\texttt{#1}}
\newcommand{\ttb}[1]{\pmb{\texttt{#1}}}
\newcommand{\tbs}{\textbackslash}
% Macros, all must be filled out
\newcommand{\ChapNum}{17}

\begin{document}
	\title
	[Chapter \ChapNum]
	{C++ Primer Plus, 5$^\text{th}$ Edition \\
	Programming Exercises \\
	Chapter \ChapNum}

	\maketitle

	\begin{cpart}{1}
		Write a program that counts the number of characters up to the
		first \$ in input and that leaves the \$ in the input stream.
	\end{cpart}

	\begin{cpart}{2}
		Wite a program that copies your keyboard input (up to the simulated
		end-of-file) to a file named on the command line.
	\end{cpart}

	\begin{cpart}{3}
		Write a program that copies one file to another.
		Have the program take the filenames from the command line.
		Have the program report if it cannot open a file.
	\end{cpart}

	\begin{cpart}{4}
		Write a program that opens two text files for input and one for
		output.
		The program should concatenate the corresponding lines of the
		input files, use a space as a separator,
		and write the results to the output file.
		If one file is shorter than the other, the remaining
		lines in the longer file should also be copied to the output file.
		For example, suppose the first input file has these contents.

		{\ttfamily
			eggs kites donuts \\
			balloons hammers \\
			stones
		}

		And suppose the second input file has these contents:

		{\ttfamily
			zero lassitude \\
			finance drama
		}

		The resulting file woul dhave these contents: 
		
		{\ttfamily
			eggs kites donuts zero lassitude \\
			balloons hammers finance drama \\
			stones
		}
	\end{cpart}

	\begin{cpart}{5}
		Mat and Pat want to invite their friends to a party, much as
		they did in Programming Exercise 8 in Chapter 16, except now they
		want a program that uses files.
		They have asked you to write a program that does the following:
		\begin{itemize}
			\item
				Reads a list of Mat's friends' names from a text file 
				called \ttt{mat.dat}, which lists one friend per line.
				The names are stored in a container and then displayed
				in sorted order.
			\item
				Reads a list of Pat's friends names from a text file called
				\ttt{pat.dat}, which lists one friend per line.
				The names are stored in a container and then displayed 
				in sorted order.
			\item
				Merges the two lists, eliminating duplicates, and stores
				the result in the file \ttt{matnpat.dat}, one friend
				per line.
		\end{itemize}
	\end{cpart}

	\begin{cpart}{6}
		Consider the class definitions of Programming Exercise 5 in
		Chapter 14.
		If you haven't yet done that exercise, do so now.
		Then do the following: 

		Write a program that uses standard C++ I/O and file I/O in 
		conjunction with data of types \ttt{employee}, \ttt{manager}, 
		\ttt{fink}, and \ttt{highfink}, as defined in Programming 
		Exercise 5 in Chapter 14.
		The program should be along the general lines of Listing 17.17 
		in that it should let you add new data to a file.
		The first time through, the program should solicit
		data from the user, show all the entries, and save the 
		information in a file.
		On subsequent uses, the program should first read and display
		the file data, let the user add data, and show all the data.
		One difference is that data should be handled by an array
		of pointers to type \ttt{abstr\_emp}.
		That way, a pointer can point to an \ttt{abstr\_emp} object
		or to objects of any of the three derived types.
		Keep the array small to facilitate checking the program;
		for example, you might limit the array to 10 elements.

		{\ttfamily
			const int MAX = 10; \hspace{8 ex} // no more than 10 objects \\
			\ldots
			abstr\_emp * pc[MAX];
		}

		For keyboard entry, the program should use a menu to offer the 
		user the choice of which type of ojbect to create.
		The menu should use a swtich to use \ttt{new} to create
		an object of the desired type and to assign the object's address
		to a pointer in the \ttt{pc} array.
		Then that object can use the virtual \ttt{setall()} function to 
		elicit the appropriate data from the user.
	
		\ttt{pc[i]->setall(); \quad 
				// invokes function corresponding to type of object}

		To save the data to a file, devise a virtual \ttt{writeall()}
		function for that purpose.

		{\ttfamily
			for(i = 0; i < index; i++) \\
			\phantom{for(}pc[i]->writeall(fout); \\
			\phantom{for(}// fout ofstream connected to output file
		}
		
		The tricky part is recovering the data from the file.
		The problem is, how can the program know wheter the next item
		to be recovered is type \ttt{employee} object, a \ttt{manager}
		object, \ttt{fink} object, or a \ttt{highfink} type?
		One approach is, when writing the data for an object file,
		precede the data with an integer that indicates the type of 
		object to follow.
		Then, on file input, the program can read the inteer and then 
		use \ttt{switch} to create the appropriate object to 
		receive the data:
		{\ttfamily
			\begin{tabbing}
				\phantom{\qquad}\=\phantom{\qquad}\=\phantom{\qquad}\= \\
				enum classkind\{Employee, Manager, Fink, Highfink\};
					// in class header \\
				\ldots \\
				int classtype; \\
				while((fin >> classtype).get(ch))\{
					// newline separates int from data \\
				\> 	switch(classtype \{ \\
				\> \> 		case Employee : pc[i] = new employee; \\
				\> \> 		\phantom{case Employee : }break;
			\end{tabbing}
		}
		Then you can use the pointer to invoke the virtual \ttt{getall()}
		function to read the information: 

		\ttt{pc[i++]->getall();}
	\end{cpart}

	\begin{cpart}{7}
		Here is part of a program that reads keyboard input into a
		vector of \ttt{string} objects, stores the string contents
		(not the objects) in a file, and then copies the file 
		contents back into a vector of \ttt{string} objects: 
		{\ttfamily
			\begin{tabbing}
				\phantom{\qquad}\=\phantom{\qquad}\=\phantom{\qquad}\= \\
				int main() \\
				\{ 
				\+ \\
					using namespace std; \\
					vector<string> vostr; \\
					string temp; \\
					\\
				\< // acquire strings \\
					cout << "Enter strings (empty line to quit):\tbs n"; \\
					while (getline(cin,temp) \&\& temp[0] != '\tbs 0') \\
					\> 	vostr.push\_back(temp); \\
					cout << "Here is your input.\tbs n"; \\
					for\_each(vostr.begin(), vostr.end(), ShowStr); \\
					\\
				\< // store in a file \\
					ofstream fout("strings.dat", ios\_base::out | 
						ios\_base::binary); \\
					for\_each(vostr.begin(), vostr.end(), Store(fout)); \\
					fout.close(); \\
					\\
				\< // recover file contents \\
					vector<string> vistr; \\
					ifstream fin("strings.dat", ios\_base::in |
						ios\_base::binary); \\
					if (!fin.is\_open()) \\
					\{ \\
					\> 	cerr << "Could not open file for input.\tbs n"; \\
					\> 	exit(EXIT\_FAILURE); \\
					\} \\
					GetStrs(fin, vistr); \\
					cout << "\tbs nHere are the strings read from the file:
						\tbs n"; \\
					for\_each(vistr.begin(), vistr.end(), ShowStr); \\
					\\
					return 0; \\
				\< \}
			\end{tabbing}
		}
	\end{cpart}
	\newpage
	\begin{cpartContinued}{7}
		Note that the file is opened in binary format and that the
		intention is that I/O be accomplished with \ttt{read()} and
		\ttt{write()}.
		Quite a bit remains to be done: 
		\begin{itemize}
			\item
				Write a \ttt{void ShowStr(const string \&)} function
				that displays a \ttt{string} object followed by
				a newline character. \\
			\item
				Write a \ttt{Store} functor that writes string information
				a file. 
				The \ttt{Store} constructor should specify an \ttt{ifstream}
				object, and the overloaded \ttt{operator()(const string \&)}
				should indicate the string to write.
				A workable plan is to first write the string's size to
				the file and then write the string's contents.
				For example, if \ttt{len} holds the string size, you could
				use this:
				{\ttfamily
					\begin{tabbing}
						\hspace{9cm}\= \\
						os.write((char *)\&len, sizeof(std::size\_t)); 
							\> // store length \\
						os.write(s.data(), len); 
							\> // store characters
					\end{tabbing}
				}
				\phantom{\quad}\\
				The \ttt{data()} member returns a pointer to an array
				that holds the characters in the string.
				It's similar to the \ttt{c\_str()} member except that the
				latter appends a null character. \\
			\item
				Write a \ttt{GetStr()} function that recovers information
				from the file.
				It can use \ttt{read()} to obtain the size of a string
				and then use a loop to read that many characters from the
				file, appending them to an initially empty temporary string.
				Because a string's data is private, you have to use a
				class method to get data into the string rather than
				read directly into it.
		\end{itemize}
	\end{cpartContinued}

\end{document}

regarding tabbing environments:
\= (set tab)
\> (advance to next tab stop)
\<
\+ (indent; move margin right)
\- (unindent; move margin left)
\'
\`
\\ (end of line; newline)
\kill (ignore preceding text; use only for spacing)

use \hspace{...} if you prefer

		{\ttfamily
			\begin{tabbing}
				\phantom{\qquad}\=\phantom{\qquad}\=\phantom{\qquad}\= \\
		
			\end{tabbing}
		}












\documentclass{amsart}

\usepackage{amssymb,latexsym, verbatim}
\thispagestyle{empty}
\pagestyle{empty}

\begin{document}
\begin{center}
	\Large {\bfseries
	\emph{C++ Primer Plus, $5^{\text{th}}$ Edition} by Stephen Prata \\
	Chapter 12: Classes and Dynamic Memory Allocation \\
	Review Questions} \normalsize \vspace{5ex}
\end{center}

% Note: in order to really get the formatting I want, I need to create my own environment. It would be similar to the enumerate environment, but instead of enclosing the enumerator in parentheses, there would just be a period after it. For example, rather than (1), we would have 1. instead.

\vfill
\noindent 1. 
\begin{minipage}[t]{11.5 cm}
	Suppose a \texttt{String} class has the following private members:
	\begin{verbatim}
		class String
		{
		private:
		    char * str;   // point to string allocated by new
		    int len;      // holds length of string
		//...
		};
	\end{verbatim}
\end{minipage} \\[1ex]
\phantom{1. }a.
\begin{minipage}[t]{11.5 cm}
	What's wrong with this default constructor?
	\begin{verbatim}
		String::String() {}
	\end{verbatim}
	\\[1ex]
	{\slshape 
		The variables \verb+str+ and \verb+len+ are not initialized.
		Also, no memory is allocated dynamically, so if the destructor
		includes the \verb+delete+ command, the program will have
		memory problems since freeing unallocated memory has
		undefined behavior.
	} \\
	{} % space for comments to be included
\end{minipage} \\[1ex]
\phantom{1. }b.
\begin{minipage}[t]{11.5 cm}
	What's wrong with this constructor?
	\begin{verbatim}
		String::String(const char * s)
		{
		    str = s;
		    len = strlen(s);
		}
	\end{verbatim}
	\\[1ex]
	{\slshape 
		\verb+str+ is a pointer to a \verb+char+ and the actual
		argument to the function is passed by value, which means
		that \verb+str+ is set to point to an automatic variable's
		memory location.
		When the function terminates, \verb+str+ will be a dangling
		pointer.
		Also, the class specified that \verb+str+ would point to
		dynamic memory, which is not the case here.
	}\\
	{} % space for comments to be included
\end{minipage} \\[1ex]
\phantom{1. }c.
\begin{minipage}[t]{11.5 cm}
	What's wrong with this constructor?
	\begin{verbatim}
		String::String(const char * s)
		{
		    strcpy(str, s);
		    len = strlen(s);
		}
	\end{verbatim}
	\\[1ex]
	{\slshape 
		We never defined the memory location of \verb+str+. 
		When we use \verb+strcpy()+, we are copying the string
		to a pointer that was never initialized.
		Also, the class specified that \verb+str+ would point to
		dynamic memory, which is not the case here.
	} \\
	{} % space for comments to be included
\end{minipage}
\vfill
\newpage

\phantom{\quad}
\vfill
\noindent 2. 
\begin{minipage}[t]{11.5 cm}
	Name three problems that may arise if you define a class in which a pointer member is initialized by using \texttt{new}. Indicate how they can be remedied. 
\end{minipage} \\[1ex]
\phantom{2. } 
\begin{minipage}[t]{11.5 cm}
	{\slshape 
		\begin{enumerate}
			\item If the memory is not deallocated then there will be
			a memory leak.
			To remedy this, you must make sure to include the 
			\verb+delete+ in the destructor.
			\item The default copy constructor will copy member by member
			and when the temporary object is destroyed, it will free 
			the memory you previously allocated to another object.
			To remedy this, you should explicitly define the copy constructor
			so that there is deep copying
			\item The default assigment operator does not do deep copying
			and will cause memory problems when the destructor is called
			for both objects. 
			To remedy this, you should explicitly overload 
			the assignment operator to allow for deep copying.
		\end{enumerate}
	} 
\end{minipage} 
\vfill

\noindent 3. 
\begin{minipage}[t]{11.5 cm}
	What class methods does the complier generate automatically if you don't provide them explicitly? Describe how these implicitly generated functions behave.
\end{minipage} \\[1ex]
\phantom{3. } 
\begin{minipage}[t]{11.5 cm}
	{\slshape 
		The compiler automatically generates a default constructor,
		destructor, copy constructor, and an overloaded assignment
		operator.
		The default constructor creates the object but does not initialize
		any variables.
		The default destructor destroys the object but does nothing else.
		The default copy constructor creates a temporary object
		whose member values are exactly those of the object used as
		an argument.
		The copy constructor uses the same destructor as the other
		objects do.
		The default overloaded assignment operator copies member by
		member, which is also shallow copying. 
	} 
\end{minipage} 
\vfill 
\newpage

\phantom{\quad}
\vfill
\noindent 4. 
\begin{minipage}[t]{11.5 cm}
	Identify and correct the errors in the following class declaration:
	\begin{verbatim}
		class nifty
		{
		// data
		    char personality[];
		    int talents;
		// methods
		    nifty();
		    nifty(char * s);
		    ostream & operator << (ostream & os, nifty & n);
		}

		nifty:nifty()
		{
		    personality = NULL; 
		    talents = 0;
		}

		nifty:nifty(char * s)
		{
		    personality = new char [strlen(s)];
		    personality = s;
		    talents = 0;
		}

		ostream & nifty:operator<<(ostream & os, nifty & n)
		{
		    os << n;
		}
	\end{verbatim}
\end{minipage}
\vfill
\newpage

\phantom{\quad}
\vfill
\phantom{2. } 
\begin{minipage}[t]{11.5 cm}
	{\slshape 
		Here is what the code should look like:
	}
	\begin{verbatim}
		using std::ostream;
		class nifty
		{
		// data
		    char * personality;
		    int talents;
		// methods
		    nifty();
		    nifty(const char * s);
		    friend ostream & operator<<(ostream & os, const nifty & n);
		};

		nifty::nifty()
		{
		    personality = NULL; 
		    talents = 0;
		}

		nifty::nifty(const char * s)
		{
		    personality = new char[std::strlen(s) + 1];
		    std::strcpy(personality, s);
		    talents = 0;
		}

		ostream & operator<<(ostream & os, const nifty & n)
		{
		    os << n;
		    return os;
		}
	\end{verbatim}
\end{minipage} 
\vfill
\newpage

\phantom{\quad}
\vfill
\noindent 5. 
\begin{minipage}[t]{11.5 cm}
	Consider the following class declaration:
	\begin{verbatim}
		class Golfer
		{
		private:
		    char * fullname; // points to string containing golfer's name
		    int games;       // holds number of holf games played
		    int * scores;    // points to first element of array of golf scores
		public:
		    Golfer();
		    Golfer(const char * name, int g = 0);
		    // creates empty dynamic array of g elements if g > 0
		    Golfer(const Golfer & g);
		    ~Golfer();
		};
	\end{verbatim}
\end{minipage} \\[1ex]
\phantom{1. }a.
\begin{minipage}[t]{11.5 cm}
	What class methods would be invoked by each of the following statements?
	\begin{verbatim}
		Golfer nancy;                         // #1
		Golfer lulu("Little Lulu");           // #2
		Golfer roy("Roy Hobbs", 12);          // #3
		Golfer * par = new Golfer;            // #4
		Golfer next = lulu;                   // #5
		Golfer hazzard = "Weed Thwacker";     // #6
		*par = nancy;                         // #7
		nancy = "Nancy Putter";               // #8
	\end{verbatim}
	\\[1ex]
	{\slshape 
		Statement #1 would use the default constructor.

		Statement #2 would use the explicit constructor where the second
		argument assumes its default value of 0.

		Statement #3 would use the explicit constructor.

		Statement #4 uses the default constructor.

		Statement #5 uses the copy constructor, and possibly the 
		default assignment operator. 

		Statement #6 uses the explicit constructor.

		Statement #7 uses the default overloaded assignment operator.

		Statement #8 uses the explicit constructor and the
		default assignment operator.
	} \\
	{} % space for comments to be included
\end{minipage} \\[1ex]
\phantom{1. }b.
\begin{minipage}[t]{11.5 cm}
	Clearly, the class requires several more methods to make it useful. What additional method does it require to protect against data corruption? \\[1ex]
	{\slshape 
		We need to explicitly define a copy constructor and
		the assignment operator to ensure deep copying. 
		Also, we must ensure that the destructor
		deallocates memory with the \verb+delete+ command. 
	}
	{} % space for comments to be included
\end{minipage} 
\vfill

\end{document}

Here is the format for questions that include multiple parts:

\noindent X. 
\begin{minipage}[t]{11.5 cm}
	The question
\end{minipage} \\[1ex]
\phantom{1. }a.
\begin{minipage}[t]{11.5 cm}
	part a \\[1ex]
	{\slshape The answer.} \\
	{} % space for comments to be included
\end{minipage} \\[1ex]
\phantom{1. }b.
\begin{minipage}[t]{11.5 cm}
	part b \\[1ex]
	{\slshape The answer}\\
	{} % space for comments to be included
\end{minipage} \\[1ex]
\phantom{1. }c.
\begin{minipage}[t]{11.5 cm}
	part c \\[1ex]
	{\slshape The answer} \\
	{} % space for comments to be included
\end{minipage} \\[1ex]
\phantom{1. }d.
\begin{minipage}[t]{11.5 cm}
	part d \\[1ex]
	{\slshape The answer} \\
	{} % space for comments to be included
\end{minipage}
\\[2ex]

\noindent X. 
\begin{minipage}[t]{11.5 cm}
	The question
\end{minipage} \\[1ex]
\phantom{3. } 
\begin{minipage}[t]{11.5 cm}
	{\slshape The answer.} 
\end{minipage} 
\\[2ex]

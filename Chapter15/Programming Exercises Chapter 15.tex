\documentclass[10 pt]{amsart}

\usepackage{amssymb,latexsym}
\usepackage{graphicx, setspace, enumerate}
% the setspace package allows us use of the 
% spacing environment (\begin{spacing}{second arg}) where
% second arg is a number to multiply to the spacing factor.
% use 2 for double space, 1 for single space, etc.

% For the cpart environment, although it would probably be better in the
% future to implement this with a list environment.
\newlength{\cgap}
\settowidth{\cgap}{\textbf{x. }}
\newlength{\cwidth}
\setlength{\cwidth}{\textwidth}
\addtolength{\cwidth}{-\cgap}
\newenvironment{cpart}[2][\cwidth]
	{%		
		\\ %
		\textbf{#2. }%
		\begin{minipage}[t]{#1}%
		\setlength{\parindent}{0pt}%
		\setlength{\parskip}{2ex}%
	}
	{%
		\end{minipage}%
	}
\newenvironment{cpartContinued}[2][\cwidth]
	{%		
		\\ %
		\textbf{#2. (continued)}%
		\\
		\phantom{#2. }
		\begin{minipage}[t]{#1}%
		\setlength{\parindent}{0pt}%
		\setlength{\parskip}{2ex}%
	}
	{%
		\end{minipage}%
	}


% set paragraph spacing like that in the book
\setlength{\parindent}{0pt}
\setlength{\parskip}{2ex}

% these new commands will make typing different formats easier.
\newcommand{\ttt}[1]{\texttt{#1}}
\newcommand{\ttb}[1]{\pmb{\texttt{#1}}}
\newcommand{\tbs}{\textbackslash}
% Macros, all must be filled out
\newcommand{\ChapNum}{X}

\begin{document}
	\title
	[Chapter \ChapNum]
	{C++ Primer Plus, 5$^\text{th}$ Edition \\
	Programming Exercises \\
	Chapter \ChapNum}

	\maketitle

	\begin{cpart}{1}
		Modify the \ttt{Tv} and \ttt{Remote} classes as follows: 
		\begin{enumerate}[a.]
			\item
				Make them mutual friends.
			\item
				Add a state variable member to the \ttt{Remote} class
				that describes whether the rmote control is in normal
				or interactive mode.
			\item
				Add a \ttt{Remote} method that displays the mode. 	
			\item
				Provide the \ttt{Tv} class with a method for
				toggling the new \ttt{Remote} member.
				This method should work only if the TV is in the on
				state.
		\end{enumerate}
		Write a short program that tests these new features.
	\end{cpart}

	\begin{cpart}{2}
		Modify Listing 15.11 so that the two exception types are classes
		derived from the \ttt{logic\_error} class provided in the
		\ttt{<stdexcept>} header file.
		Have each \ttt{what()} method report the function name
		and the nature of the problem.
		The exception object need not hold the bad values;
		they should just support the \ttt{what()} method.
	\end{cpart}

	\begin{cpart}{3}
		This exercise is the same as Programming Exercise 2, except that
		the exceptions should be derived from a base class (itself
		derived from \ttt{logic\_error}) that stores the two
		argument values, the exceptions should have a method that reports
		these values as well as the function name, and a single
		\ttt{catch} block that catches the base-class exemption should
		be used for both exceptions, with either exception causing 
		the loop to terminate.
	\end{cpart}

	\begin{cpart}{4}
		Listing 15.16 uses two \ttt{catch} blocks after each \ttt{try}
		block so that the \ttt{nbad\_index} exception leads to the 
		\ttt{label\_val()} method being invoked.
		Modify the program so that it uses a single \ttt{catch}
		block after each \ttt{try} block and uses
		RTTI to handle invoking \ttt{label\_val()} only when appropriate.
	\end{cpart}

\end{document}

regarding tabbing environments:
\= (set tab)
\> (advance to next tab stop)
\<
\+ (indent; move margin right)
\- (unindent; move margin left)
\'
\`
\\ (end of line; newline)
\kill (ignore preceding text; use only for spacing)

use \hspace{...} if you prefer

		{\ttfamily
			\begin{tabbing}
				\phantom{\qquad}\=\phantom{\qquad}\=\phantom{\qquad}\= \\
		
			\end{tabbing}
		}











